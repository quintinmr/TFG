\documentclass{article}
\begin{document}

\title{Notas}

\begin{itemize}
    \item \textbf{¿Qué son los datos ómicos?}
    En el campo de la Biología, se ha introducido el término "ómicas" para hacer referencia a distintos métodos de 
    obtención de información. Engloba a todas las disciplinas de la biología con nombre acabado en el sufijo "-ómica":
    genómica, proteómica, metabolómica, metagenómica, fenómica y transcriptómica, entre otras.
    
    A la información extraída de las células, por medio de tecnologías usadas en las ciencias ómicas, es lo que se 
    conoce como "dato ómico". Un dato ómico puede verse como un tipo particular de datos de alta dimensión.

    \item \textbf{Estructura de los datos ómicos}



    1 Los datos de alta dimensión se refieren a conjuntos de datos con una gran cantidad de características
    o variables. 

\end{itemize}

\end{document}
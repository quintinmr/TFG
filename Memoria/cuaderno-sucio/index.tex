\documentclass{article}

\begin{document}

\section*{Prototipo de Índice}

\begin{itemize}
    \item[1] Estado del arte
    \item[2] Objetivos
    \item[3] Introducción
    \item[4] Datos ómicos
    \begin{itemize}
        \item[3.1] ¿Qué son los datos ómicos?
        \item[3.2] Estructura de los datos ómicos
        \item[3.3] Trasnscriptómica (¿espacial?)
    \end{itemize}
    \item[4] Fundamentos matemáticos
    \begin{itemize}
        \item[4.1] Técnicas multivariantes (aplicables al tratamiento de datos ómicos)
        \item ...¿?
    \end{itemize}
    \item[5] Fundamentos informáticos
    \begin{itemize}
        \item[5.1] Fuentes de datos ómicos (bases de datos como GEO, etc.)
        \item[5.2] Estructuras de almacenamiento (microarrays)
        \item[5.3] Tratamiento de datos ómicos (preprocesamiento, exploración y visualización, PCA, etc.)
        \item[5.4] Implementación (python/R) de técnicas multivariantes para el manejo de E.D ómicos
        \item[5.5] Técnicas de aprendizaje automático para la identificación de patrones y clasificación de datos ómicos.
        \item[5.6] Estadística estacial en transcriptómica ¿¿?? (desarrollo de módulos programáticos para datos ómicos con
        dependencia espacial)¿?
    \end{itemize}
    \item[6] Aplicación (realista y funcional (sin usar las bases de datos convencionales (muy "trilladas")))
    \item[7] Resultados y conclusiones
\end{itemize}

\end{document}
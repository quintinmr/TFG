\documentclass{article}
\usepackage{setspace}
\begin{document}

\title{Notas}

\begin{itemize}
    \item \textbf{¿Qué son los datos ómicos?}

    A mediados del siglo XX, se da un giro crucial en la comprensión y el estudio de los organismos vivos con
    la llegada de la \textbf{biología molecular}. Este avance trajo consigo el desarrollo de herramientas que
    facilitaron el análisis de los diversos y complejos niveles de estos organismos, generando grandes volúmenes
    de datos en periodos de tiempo relativamente cortos: las ciencias ómicas.

    Las ciencias ómicas abarcan diversas áreas de la biología cuyo nombre termina con el sufijo "-ómica", tales
    como la genómica, proteómica, metabolómica, metagenómica, fenómica y transcriptómica, entre otras.
    
    La información obtenida a través de tecnologías aplicadas en las ciencias ómicas se denomina \textbf{"dato ómico"}.
    Este tipo de dato representa un conjunto de información caracterizado por su alta dimensionalidad, lo que lo
    convierte en una herramienta clave para el análsis integral de organismos vivos.

    Aldededor del 1980, se acuñó el término "ómica" para hacer referencia al estudio de un conjunto de moléculas. Por
    ejemplo, la genómica se refiere al estudio de un conjunto de genes en el ADN, la transcriptómica es el estudio de
    muchos transcritos o ARNm, o la proteómica; el estudio de un conjunto de proteinas, etc.
    
    
    \item \textbf{Estructura de los datos ómicos}



    1 Los datos de alta dimensión se refieren a conjuntos de datos con una gran cantidad de características
    o variables. 

\end{itemize}


https://www.institutoroche.es/static/archivos/Informes_anticipando_CIENCIAS_OMICAS.pdf?utm_source=chatgpt.com 14/01/2025
https://revista.unam.mx/vol.18/num7/art54/index.html?utm_source=chatgpt.com  14/01/2025
https://www.researchgate.net/publication/338777784_Biological_Pattern_Formation biological patterns 14/01/2025
https://omics.org/History_of_Omics historia ciencias ómicas 14/01/2025
\end{document}
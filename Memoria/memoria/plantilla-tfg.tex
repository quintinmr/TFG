\documentclass[print, color, oneside]{ugrTFG}
\usepackage{url}           % Para poder manejar enlaces URL correctamente
\usepackage[utf8]{inputenc}
\usepackage{amsmath}       % Si usas matemáticas
\usepackage{geometry}
\usepackage{tcolorbox}
\usepackage{listings}
\usepackage{csquotes}
\usepackage[perpage]{footmisc} % Reinicia la numeración en cada página
\usepackage[backend=biber,style=numeric,sorting=none]{biblatex}
\addbibresource{../capitulos/referencias.bib}
\geometry{twoside=false}
\geometry{left=2.5cm, right=2.5cm, top=3cm, bottom=3cm}

\lstset{
    breaklines=false,   % Evita que las líneas se corten
    prebreak={},        % Evita cortes de línea innecesarios
    xleftmargin=-2.2cm,   % Desplaza el código a la izquierda
    numbers=none,       % Quita la numeración de líneas
    frame=none          % Quita el recuadro alrededor del código
}

% -------------------------------------------------------------------
% INFORMACIÓN DEL TFG Y EL AUTOR
% -------------------------------------------------------------------

\newcommand{\miTitulo}{Metodologías Multivariantes para la Identificación de Patrones Biológicos\xspace}
\newcommand{\miNombre}{Quintín Mesa Romero\xspace}
\newcommand{\miGrado}{Doble Grado en Ingeniería Informática y Matemáticas}
\newcommand{\miFacultad}{Facultad de Ciencias}
\newcommand{\miFacultadBis}{E.T.S. Ingenierías Informática y de Telecomunicación}
\newcommand{\miUniversidad}{Universidad de Granada}

% Añadir tantos tutores como sea necesario separando cada uno de ellos mediante el comando `\medskip` y una línea en blanco
\newcommand{\miTutor}{
José Luis Romero Béjar \\ \emph{Departamento de Estadística e Investigación Operativa}
}

\newcommand{\miCurso}{2024\-2025\xspace}

\hypersetup{
    pdftitle={\miTitulo},
    pdfauthor={\textcopyright\ \miNombre, \miFacultad, \miFacultadBis, \miUniversidad}
}

\begin{document} 

\newtcolorbox{mybox}[1]
{colback=black!5!white,colframe=black!75!black, width=1.05\linewidth,fonttitle=\bfseries,title=#1}

\maketitle

% -------------------------------------------------------------------
% FRONTMATTER
% -------------------------------------------------------------------
%frontmatter

\newpage
\input{../preliminares/declaracion-originalidad}   
%\input{preliminares/dedicatoria}                % Opcional
%\input{preliminares/agradecimientos}            % Opcional

\chapter{Introducción}

La biología, como disciplina científica, ha experimentado una evolución notable en las últimas décadas, 
pasando de enfoques cualitativos y descriptivos a un análisis más detallado y cuantitativo de los organismos 
vivos. Este cambio de paradigma se produjo a mediados del siglo XX con la llegada de la \textit{biología molecular},
tras casi dos siglos de preeminencia del naturalismo basado en la observación y la contemplación. 
Este avance marcó el inicio de una nueva era en la que el desarrollo de ciertas herramientas tecnológicas permitió 
analizar los diversos y complejos niveles de organización de los organismos, generando grandes volúmenes de datos 
en periodos relativamente cortos: la \textit{era de las ciencias ómicas}\cite{referencia-1}\cite{referencia-2}. \newline%\cite{referencia1}\cite{referencia2}. \newline

En este contexto, las ciencias ómicas surgieron como un marco integrador que engloba el conocimiento derivado 
de la aplicación de tecnologías avanzadas para el estudio a nivel molecular de los distintos elementos que 
conforman los sistemas biológicos, como células, tejidos e individuos. Estas disciplinas no solo permiten 
analizar la complejidad interna de los organismos, sino también comprender las interacciones dinámicas 
entre sus componentes internos y los factores externos con los que estos interactúan. Ofrecen, en definitiva,
una perspectiva holística del individuo, proporcionando una visión detallada del funcionamiento de sus células 
y de la influencia del entorno que las rodea\cite{referencia-2}. \newline% \cite{referencia2}. \newline

El término \textit{ómica} fue acuñado en la década de 1980 para referirse al estudio de 
\textit{conjuntos de moléculas} específicas, como genes (genómica), transcripciones de ARN (transcriptómica), 
proteínas (proteómica) o metabolitos (metabolómica), entre otros. Estas disciplinas han evolucionado 
significativamente gracias a los avances tecnológicos que permiten abordar la complejidad inherente de los 
sistemas biológicos analizados. De hecho, este es el máximo distintivo de las ciencias ómicas: el uso de las 
llamadas \textit{tecnologías ómicas}, herramientas de alto rendimiento diseñadas para generar grandes cantidades 
de datos en un solo experimento a partir de una única muestra. Este enfoque masivo en la obtención de datos, 
conocido como \textit{Big Data}, ha transformado profundamente el análisis biológico, permitiendo explorar 
dinámicas moleculares con un gran nivel de detalle.\cite{referencia-2}\cite{referencia-5}\newline% \cite{referencia2} \cite{referencia5}. \newline

La integración de las ciencias ómicas con metodologías avanzadas de análisis, como las técnicas multivariantes 
y el aprendizaje automático, ha marcado un hito en la investigación biomédica, abriendo nuevas fronteras en la 
comprensión de los complejos sistemas biológicos. Estas metodologías, que permiten gestionar y analizar grandes 
volúmenes de datos con múltiples dimensiones, son fundamentales para descubrir patrones biológicos subyacentes 
que, de otro modo, podrían pasar desapercibidos utilizando métodos tradicionales\cite{tarca}. Técnicas multivariantes, como 
el análisis de componentes principales (PCA), el análisis clúster, el análisis factorial o el análisis discriminante, 
facilitan la identificación de relaciones y la reducción de la dimensionalidad en los datos, lo que es crucial para 
poder extraer información relevante de los voluminosos conjuntos de datos generados\cite{intro-1}.\newline

A medida que los volúmenes de datos generados por las tecnologías ómicas se incrementan, la \textit{bioinformática} se ha 
consolidado como una disciplina esencial para procesar, gestionar y analizar dichos datos\cite{kaneshia}. Facilita la identificación y 
visualización de patrones biológicos complejos a partir de grandes bases de datos, mediante el uso de algoritmos avanzados,
herramientas computacionales y modelos estadísticos. Este enfoque es fundamental para descubrir asociaciones moleculares, 
determinar biomarcadores relevantes y comprender las bases genéticas de enfermedades\cite{koh}. En este sentido, las herramientas
bioinformáticas, como los lenguajes de programación R y Python, entre otros, junto con plataformas especializadas como Bioconductor, 
permiten realizar análisis profundos de datos ómicos a gran escala, proporcionando los recursos necesarios para un 
manejo efectivo y preciso de la información biológica\cite{bioconductor-1}. \newline

Además, la combinación de las ciencias ómicas con estas metodologías avanzadas ha mejorado significativamente 
nuestra comprensión de los procesos biológicos y ha facilitado el desarrollo de estrategias diagnósticas y terapéuticas 
innovadoras. En particular, las técnicas multivariantes y el aprendizaje automático han demostrado ser esenciales 
para la identificación de patrones biológicos en diversas enfermedades, desde el cáncer hasta trastornos neurodegenerativos, 
así como para predecir la respuesta a distintos tratamientos. Esta integración ha impulsado el avance hacia la medicina 
personalizada y de precisión, en la que los tratamientos se ajustan a las características individuales de cada paciente, 
haciéndolos mucho más eficientes y reduciendo efectos adversos. \newline

En el presente trabajo, se explorará el uso de la transcriptómica, como ciencia ómica y las metodologías avanzadas de análisis de 
datos, como las técnicas multivariantes y el aprendizaje automático, para la identificación y clasificación 
de ciertos patrones biológicos. Se realizará una revisión teórica de las técnicas multivariantes más comunes, 
anteriormente mencionadas, aunque nos centraremos en una de ellas con el fin de proporcionar una base sólida 
para su aplicación práctica en datos ómicos. Posteriormente, se llevará a cabo una implementación realista y 
funcional para el análisis de datos biológicos, aplicando técnicas de aprendizaje automático para la identificación 
de patrones biológicos significativos. A través de estas metodologías avanzadas, se intentará simplificar los datos 
ómicos para poder extraer la información clave que permita clasificar y entender mejor los patrones biológicos, 
mejorando así la precisión de los modelos predictivos.


\thispagestyle{empty}
\vspace*{\fill}
\begin{center}
    \large Parte I \\
    \vspace{0.5cm}           
    \LARGE \textbf{DATOS ÓMICOS}
\end{center}
\vspace*{\fill}
\newpage
\setcounter{page}{1}  % Opcional: reiniciar la numeración de páginas si lo necesitas

\newpage


\section{Datos ómicos}

A la información cuantitativa y cualitativa obtenida a partir de las tecnologías utilizadas
en las distintas ciencias ómicas, se le denomina \textbf{datos ómicos}. Estos datos abarcan 
información genética (genómica), de expresión génica (transcriptómica), de proteínas (proteómica),
metabolitos (metabolómica) y otras áreas emergentes dentro de las ciencias ómicas.

Una de sus características más relevantes es su \textbf{alta dimensionalidad}, lo que genera conjuntos de datos 
masivos y complejos. Esta naturaleza multidimensional y heterogénea de los datos ómicos presenta 
desafíos significativos en su procesamiento, análisis e interpretación.

\subsection{Estructura de los datos}

Los datos con los que trabajaremos se caracterizan por tener una estructura parecida. Analizaremos un
conjunto con pocas muestras frente al gran número de características que observaremos sobre ellas. Apreciamos
aquí el carácter de alta dimensionalidad de los datos ómicos. \newline

Las características que analizamos pueden ser de diferentes tipos, como el nivel de fluorescencia, en el caso
de que estemos trabajando con microarrays \footnote[1]{Microarray: La tecnología de microarrays permite estudiar 
la expresión de múltiples genes simultáneamente. Consiste en fijar miles de secuencias génicas en un chip de vidrio. 
Al exponer una muestra de ADN o ARN, el apareamiento de bases complementarias genera una señal luminosa medible, 
indicando los genes expresados en la muestra.}, como los de ADN, metilación o proteínas, o el número de lecturas 
alineadas obtenidas en procedimientos de secuenciación. Estas características, pueden estar asociadas a un
elemento de análisis o a un conjunto de muestras en un microarray. O bien, la información puede corresponder a
un gen, un exón\footnote[2]{Exón: un exón es una región del genoma que termina dentro de una molécula de ARN mensajero.
Algunos exones son codificantes, es decir, contienen información para fabricar una proteína, mientras que otros 
son no codificantes. Los genes del genoma están formados por exones e intrones.},
una proteína o una región específica del genoma. \newline

Denotaremos por $N$ al número de características observadas, que será un valor relativamente grande, del orden
de miles. Como hemos mencionado anteriormente, estas características se observan sobre un conjunto reducido de 
individuos, del orden de las decenas, en el mejor de los casos. Sea entonces $n$ el número de muestras sobre
las que serán observadas las variables (características). \newline

Por consiguiente, el problema se enmarca dentro del campo de la \textbf{estadística de alta dimensión}. Esta
situación, donde $N$ supera a $n$, contrasta con lo que se observa en los enfoques estadísticos convencionales,
en los cuales suele ocurrir todo lo contrario: el número de muestras es mayor que el de variables. Aunque esta
desigualdad presenta limitaciones, también abre un nuevo campo de investigación con retos que los métodos 
tradicionales no pueden resolver, lo que motiva el desarrollo de nuevos procedimientos que se explorarán más 
adelante. \newline

Las características las almacenaremos en una matriz, que llamaremos \textbf{matriz de expresión}, dada por:

\[
    x = [x_{ij}]_{i,j=1,...,n}
\]

donde $x_{ij}$ cuantifica la característica $i$ en la muestra $j$. \newline

\textbf{Nota.} Observemos que en un contexto estadístico
convencional, la matriz de datos sería la matriz transpuesta de la que vamos a estar utilizando. \newline

En el supuesto de que $x_{ij}$ esté asociado con un microarray de ADN, entonces, mide un nivel de fluorescencia,
tomando valores positivos, aunque pudiera ser que, tras el procesado de los datos, se diera lugar a expresiones 
negativas. Por su parte, si se tratase de un dato obtenido mediante la técnica de secuenciación RNA-seq, tendríamos conteos; 
número de lecturas cortas que se alinean sobre un gen, exón o una zona genómica concretos. Un mayor número de
lecturas será indicativo de una mayor expresión de dicha característica. \newline

Los valores observados de una característica sobre el conjunto de todas las muestras (una fila de la matriz de
expresión) son, en el ámbito de la transcriptómica, lo que se conoce como \textit{perfil}, o de forma más general,
perfil de expresión. \newline

En la matriz $x$ los valores correspondientes a las diferentes muestras son independientes entre sí, aunque pueden
haber sido obtenidos bajo condiciones distintas. Por lo tanto, no se trata de réplicas de una misma condición
experimental, sino de observaciones independientes. Es decir, presentan independencia condicional. Sin embargo, las
filas de $x$ representan vectores que sí están relacionados. Por ejemplo, en una matriz de expresión génica, los 
valores de expresión de las filas no son independientes, debido a que los genes tienden a actuar de manera coordinada. \newline

Por lo general, los datos en las columnas de la matriz $x$, no pueden compararse directamente entre sí, por la presencia
de diversos artefactos técnicos y ruido en la medición de la característica de interés. Es por ello que se han desarrollado
técnicas para corregir estos problemas, denominadas como \textbf{técnicas de preprocesado}. Al aplicar estos métodos, los
datos dejan de ser completamente independientes. No obstante, en la mayoría de los estudios este aspecto no se suele
tener en cuenta. Tras la normalización, los datos siguen considerandose independientes por columnas (muestras) y dependientes
por filas. \newline

A la información o variables que describen y caracterizan a las muestras, las llamaremos \textbf{metadatos} o 
\textbf{variables fenotípicas}. En este contexto, el uso de este término es adecuado porque estas variables reflejan
atributos medibles y observables de las muestras, lo que se conoce en el ámbito de la biología como \textit{fenotipo}. 
Normalmente tendremos varias variables fenotípicas. 
Llamaremos $y = (y_{1},...,y_{n})$ a los valores observados de una variable en las $n$ muestras. Uno de los casos más típicos
de variable fenotípica es cuando se tienen dos grupos de muestras: casos (individuos que tienen la enfermedad) y controles
(no tienen la enfermedad o condición de interés). En este caso tendríamos $y_{i} = 1$, para un caso e $y_{i} = 0$, si es control.
Si tuvieramos la situación en la que hay $k$ grupos a comparar, con $k > 2$, entonces se utiliza $y_{i} \in \{1,...,n\}$ 
con $i = 1,...,n$. Hemos de recalcar que los valores $y_{k}$ son arbitrarios y pueden tomar cualquier otro par de valores. \newline


% toda la información obtenida del libro "G. Ayala - Bioinformática Estadística (2023).pdf"
% definición de microarray: https://www.genome.gov/es/genetics-glossary/Tecnolog%C3%ADa-de-microarrays-chips-de-ADN-o-ARN 
% definición de axon: https://www.genome.gov/genetics-glossary/Exon
% definición de fenotipo: https://www.genome.gov/es/genetics-glossary/Fenotipo 


\subsection{Problema Estadístico}

Normalmente, las técnicas estadísticas utilizadas en muchos campos se basan en contextos en los que 
el número de muestras, $n$, es mayor que el de variables, $N$. Sin embargo, en el caso de los datos ómicos, esta relación se invierte,
lo que obliga a ajustar estos procedimientos de manera que, en algunos casos, la adaptación resulta más o menos exitosa. En otras palabras,
la falta de suficientes muestras para la cantidad de variables presentes, hace que sea extremadamente difícil encontrar un modelo que pueda capturar
de manera precisa la relación entre las variables predictores y la variable respuesta. Esto se debe a que no tenemos una cantidad adecuada 
de datos para entrenar de manera efectiva un modelo estadístico que pueda generalizarse de manera fiable a nuevas observaciones.

La dificultad de analizar datos de alta dimensionalidad resulta además de la conjunción de dos efectos.

En primer lugar, los espacios de alta dimensión tienen propiedades geométricas que son contra-intuitivas y alejadas de las propiedades
que se pueden observar en espacios bidimensionales o tridimensionales. 

En segundo lugar, las herramientas de análisis de datos suelen diseñarse teniendo en cuenta propiedades intuitivas
y ejemplos en espacios de baja dimensión; por lo general, las herramientas de análisis de datos se ilustran mejor en espacios de dos y 
tres dimensiones, por razones obvias. El problema es que esas herramientas también se utilizan cuando los datos son de alta dimensión y
más complejos. En este tipo de situaciones, perdemos la intuición del comportamiento de las herramientas y podemos sacar conclusiones erróneas
sobre sus resultados, dificultando la construcción de modelos estadísticos precisos. \newline

Por todo ello, en este contexto, las técnicas estadísticas utilizadas son mera aplicación de procedimientos diseñados para la situación antes comentada en la 
que el número de muestras es mayor que el de variables. \newline

Uno de los principales retos que se abordarán es el análisis de expresión diferencial, 
que examina cómo varían los niveles de expresión génica\footnote[5]{Expresión génica: La expresión génica es el proceso por el cual la información codificada por un gen se usa 
para producir moléculas de ARN que codifican para proteínas o para producir moléculas de ARN no codificantes que cumplen otras funciones. La expresión génica
actúa como un “interruptor” que controla cuándo y dónde se producen moléculas de ARN y proteínas y como un “control de volumen” para determinar qué cantidad
de esos materiales se produce} entre distintas condiciones experimentales o grupos de individuos. En particular, se
busca determinar si existe una relación entre el perfil de expresión génica y una variable fenotípica específica. Este enfoque, denominado análisis
de expresión diferencial marginal, permite explorar asociaciones entre conjuntos de genes, organizados como grupos de filas dentro de la matriz de expresión, 
y la característica fenotípica de interés. A este tipo de análisis se le conoce también como análisis de conjuntos de genes o \textit{gene set analysis}, y 
su objetivo es identificar patrones de expresión que puedan estar vinculados a determinados rasgos biológicos.

% https://www.genome.gov/es/genetics-glossary/Expresion-genica -- Definiición expresión génica


\subsection{Transcriptómica: datos RNA-seq y single-cell RNA-seq}

Entre las distintas tecnologías utilizadas en la generación de datos ómicos, la \textbf{transcriptómica} desempeña un papel fundamental en el análisis de la expresión génica, y en particular, las 
tecnologías de RNA-seq y single-cell RNA-seq han revolucionado este campo al permitir la cuantificación precisa de los 
niveles de ARN mensajero en diferentes condiciones biológicas. Estas técnicas producen datos con una estructura matricial 
compleja, caracterizada por un alto número de variables (genes) frente a un número reducido de muestras (individuos o células). 
Esta estructura plantea desafíos en términos de almacenamiento, procesamiento y análisis, debido a la alta dimensionalidad de 
los datos generados. Nos centraremos en la transcriptómica por su capacidad para ofrecer una visión dinámica 
y profunda de la actividad celular, permitiendo identificar patrones de expresión génica que reflejan procesos biológicos clave. \newline

En este apartado, se describirá la estructura de los datos obtenidos mediante RNA-seq y single-cell RNA-seq, y se discutirán los 
principales retos asociados a su manejo, desde las consideraciones técnicas hasta las implicaciones estadísticas y computacionales, 
que hacen de la transcriptómica un área idónea para aplicar metodologías multivariantes en la identificación de patrones biológicos.

\subsubsection{¿Qué es la transcriptómica? Tecnologias}

La transcriptómica es la rama de la biología que estudia el conjunto completo de ARN (ácido ribonucleico) transcritos en una
célula, tejido u organismo en un momento determinado, bajo condiciones específicas. A este conjunto se le denomina transcriptoma. Se centra en la cuantificación y caracterización
de los distintos tipos de ARN, incluyendo ARN mensajero (mARN), ARN de transferencia (tRNA), ARN ribosomal (rRNA) y ARN 
no codificante (ncRNA), entre otros. Este campo ha evolucionado significativamente desde la formulación del dogma central de la biología molecular por Francis Crick
en 1958, que estableció la transferencia de información genética desde el ADN al ARN y posteriomennte a las proteínas. \newline

A medida que la transcriptómica ha avanzado, se han ido desarrollando varias tecnologías para deducir y cuantificar el transcriptoma, basadas
tanto en hibridación como en secuenciación. Los enfoques basados en hibridación, como los microarrays, pese a que son más económicos y 
tienen un alto rendimiento, dependen del conocimiento existente sobre la sencuencia del genoma. \newline

A diferencia de los métodos basados en microarrays, los enfoques basados en secuencias determinan directamente la secuencia de ARN. Inicialmente, se utilizó
la secuenciación de Sanger de bibliotecas de ARN, pero era bastante costosa y de bajo rendimiento y generalmente no cuantitativa. Se 
desarrollaron métodos basados en etiquetas para superar estas limitaciones, pero tenían el inconveniente de que estaban basados en la 
costosa tecnolgía de secuenciación de Sanger, y una parte significativa de las etiquetas cortas no se podían asignar de forma única al
genoma de referencia. Todas estas desventajas limitan el uso de la tecnología de secuenciación tradicional para anotar la estructura de los 
transcriptomas. \newline

Tecnologías de secuenciación de ADN de alto rendimiento como \textbf{RNA-seq} y \textbf{single-cell RNA-seq} han emergido como herramientas 
clave para estudiar la expresión génica a gran escala. Estas técnicas permiten la cuantificación precisa de los niveles de ARN en diferentes 
condiciones biológicas, a diferencia de los métodos basados en microarrays, lo que proporciona información valiosa sobre la actividad celular 
y los mecanismos de regulación genética. Sin embargo, los datos obtenidos mediante RNA-seq y single-cell RNA-seq poseen características 
específicas que influyen en su representación y análisis. Estas tecnologías generan grandes volúmenes de datos con una estructura 
matricial compleja, en la que el número de características (genes) supera ampliamente al número de muestras, lo que da lugar a retos 
significativos en términos de almacenamiento, procesamiento y análisis.

\subsubsection{RNA-seq}

El método RNA-seq (secuenciación de ARN) consiste en la conversión de una muestra de ARN (total o fraccionado) en una biblioteca de ADNc 
(ADN codificado), que luego es secuenciada utilizando tecnologías de secuenciación profunda. Genera un conjunto masivo de datos que consiste 
en lecturas cortas de ARN transcrito, secuencias que generalmente varían entre 30 y 400 pares de bases de longitud, representando fragmentos 
de transcritos provenientes de ARN mensajero (ARNm) o ARN no codificante. Estas secuencias se alinean con un genoma de referencia o con 
transcritos de referencia para mapear la estructura transcripcional y cuantificar la expresión génica. Una de las principales aplicaciones 
de RNA-seq es el análisis de expresión diferencial, mencionado en la sección previa y que abordaremos de forma práctica, que permite comparar 
los niveles de expresión de genes entre diferentes condiciones 
biológicas, como células tratadas frente a no tratadas, tejidos sanos frente a cancerosos o distintos estados del desarrollo. Esto ofrece 
una visión detallada de los cambios en la actividad génica y ayuda a identificar biomarcadores, rutas metabólicas alteradas o procesos 
reguladores clave. Además, RNA-seq no está limitado a detectar solo transcritos que corresponden a secuencias genómicas conocidas, 
lo que lo hace particularmente útil para organismos no modelo o cuando se carece de un genoma de referencia bien caracterizado. \newline
% REFERENCIA: RNA-seq: A revolutionary tool....

En la secuenciación de ARN, las lecturas generadas a partir de las muestras de ARN se alinean contra un genoma de referencia o se ensamblan de
nuevo para crear un "mapa transcripcional". Si se dispone de un genoma de referencia, los datos se alinean para identificar la ubicación exacta de
los transcritos en el genoma, permitiendo la cuantificación de la expresión génica. En casos donde no hay un genoma de referencia, las lecturas 
de ARN se ensamblan para generar una secuencia de contigs \footnote[3]{Contig: Tramo de secuencia continua in silico generada por alineamiento 
de lecturas de secuencias solapantes.} que luego se pueden anotar funcionalmente. Además de mapear las lecturas a un genoma, 
se deben identificar eventos de empalme (splicing \footnote[4]{Splicing: el splicing o empalme, ocurre al final del proceso de transcripción e 
implica cortar y reorganizar secciones de ARNm.}), que es crucial para detectar variantes de splicing alternativo. Este proceso es especialmente 
importante para genes que tienen varios exones, ya que las lecturas pueden cruzar estos empalmes y revelar alternativas de splicing que no son 
evidentes con tecnologías anteriores.  % REFERENCIA: Mapping and quantifying mammalian transcriptomes by RNA-seq

% contig: https://www.institutoroche.es/recursos/glosario/contig
% splicing: https://www.yourgenome.org/theme/what-is-rna-splicing/

Pese a las ventajas que la RNA-seq tiene frente a tecnologías anteriores, los conjuntos de datos producidos son grandes y complejos y la interpretación
no es sencilla. La interpretación de los datos de secuenciación de ARN depende de la cuestión científica de interés. El objetivo principal de muchos 
estudios biológicos es el perfil de expresión génica entre muestras, que es particularmente relevante, por ejemplo, para experimentos controlados 
que comparan la expresión en cepas de tipo salvaje y mutantes del mismo tejido, comparando células tratadas versus no tratadas, cáncer versus normal, etc.
% REFERENCIA: transcriptomics-RNA-seq-3

Por otra parte, los datos RNA-seq requieren estar en unos formatos específicos para su tratamiento. Formatos adecuados para almacenar secuencias tanto
de ácidos nucleicos como de proteínas. Estos son: formato \textbf{FASTA} y \textbf{FASTQ}.

\begin{itemize}
    \item \textbf{Formato FASTA:}
    \item \textbf{Formato FASTAQ:} es el más popular y consiste en cuatro líneas por lectura:
    \begin{itemize}
        \item La primera comienza con el carácter "@" y contiene el nombre de la secuencia. Opcionalmente, puede incluir una descripción.
        \item La segunda línea contiene la secuencia con las letras correspondientes, dependiendo del tipo de secuencia del que se trate (nucleótido o aminoácido).
        \item La tercera comienza con el carácter "+" y contiene información opcional sobre la secuencia.
        \item La cuarta y última línea cuantifica la calidad o confiabilidad de cada base en la secuencia recogida en la segunda línea, basada en el índice
        \textit{Phred} y su codificación.
    \end{itemize}
\end{itemize}

\subsubsection{sigle-cell RNA-seq}

\newpage
\thispagestyle{empty}
\vspace*{\fill}
\begin{center}
    \large Parte II \\
    \vspace{0.5cm}           
    \LARGE \textbf{FUNDAMENTOS MATEMÁTICOS}
\end{center}
\vspace*{\fill}
\newpage
\setcounter{page}{1}  

\newpage

\chapter{Técnicas multivariantes: fundamentos y desarrollo del análisis clúster}

El análisis de datos en ciencias ómicas requiere metodologías capaces de manejar la complejidad 
inherente a los sistemas estudiados. En particular, las técnicas multivariantes han demostrado ser 
herramientas fundamentales para la exploración, modelado e interpretación de datos de alta dimensión. 
Estas metodologías permiten identificar relaciones entre variables, reducir la dimensionalidad y 
clasificar observaciones en función de patrones subyacentes. \newline

Este capítulo está estructurado en dos secciones. En primer lugar, se presentarán las principales 
técnicas multivariantes, destacando su utilidad y objetivos dentro del análisis de datos. Posteriormente, 
se abordará en profundidad el análisis clúster, una técnica multivariante ampliamente utilizada para 
la identificación de patrones en grandes volúmenes de datos. Su aplicación en el ámbito biológico 
permite revelar estructuras subyacentes en datos complejos, facilitando la comprensión de procesos 
como la agrupación de expresiones génicas o la clasificación de organismos en función de sus características. \newline

Dado que en el capítulo dedicado a los datos ómicos hemos introducido la matriz de datos ómicos $X$, que 
representa las $N$ características medidas sobre $n$ muestras, mantendremos esta notación en el desarrollo de
los fundamentos matemáticos sobre los que se basa este trabajo. Así, consideraremos que la matriz de expresión 
$X \in \mathbb{R}^{N \times n}$ almacena las observaciones de nuestras variables, con filas representando las
características y columnas las muestras. 



\section{Preliminares}

La información aquí recogida se ha extraído principalmente de las fuentes \cite{hist-mul-1}, \cite{hist-mul-2}, \cite{Bib-1}. \newline %history-mul-1, history-mul-2, %Bib-1

El análisis multivariante es una herramienta clave para explorar y comprender la complejidad de los sistemas 
biológicos, económicos y sociales. Su capacidad para procesar múltiples variables simultáneamente permite identificar 
patrones ocultos en grandes volúmenes de datos. \newline

Las técnicas multivariantes son fundamentales para abordar la complejidad de los datos en diversas disciplinas, 
incluyendo las ciencias ómicas, donde se requieren metodologías capaces de gestionar la alta dimensionalidad y 
variabilidad de los datos obtenidos. Estas herramientas permiten descubrir relaciones entre variables, 
reducir la dimensionalidad y clasificar observaciones, lo que facilita la interpretación y el modelado de sistemas 
complejos. \newline

El desarrollo del análisis multivariante se remonta a principios del siglo XX, cuando pioneros como Karl Pearson y R.A
Fisher introdujeron técnicas fundamentales como el análisis de componentes principales y el análisis discriminante.
Posteriormente, C.R. Rao y otros investigadores expandieron estos métodos, estableciendo bases matemáticas sólidas que
han permitido su aplicación en un amplio espectro de disciplinas. Estas técnicas han ido desarrollándose exponencialmente
con el avance de la computación, facilitando el procesamiento de grandes volúmenes de datos y dando lugar a análisis 
mucho más sofisticados en muchas áreas como la biología, la economía, las ciencias sociales, etc. \newline

En términos generales, las metodologías multivariantes pueden dividirse en dos grandes enfoques: \textit{descriptivo}
e \textit{inferencial}. El primero busca simplificar la estructura de los datos y revelar relaciones latentes entre
variables, mientras que el segundo permite realizar pruebas de hipótesis considerando múltiples variables de manera
simultánea, asegurando la validez estadística de los resultados. La elección de la técnica adecuada depende del tipo 
de datos y de la pregunta de investigación. A continuación, se presentan algunas de las principales metodologías multivariantes:

\subsection{Análisis de Componentes Principales (PCA)}

El \textit{análisis de componentes principales (PCA)} fue introducido por primera vez por Karl Pearson a principios del siglo XX. El 
tratamiento formal de esta técnica se debe a Hotelling (1933) y Rao (1964). Su propósito era facilitar la comprensión de conjuntos 
de datos complejos mediante la reducción de su dimensionalidad, minimizando la pérdida de información. En PCA, un conjunto de $N$ variables correlacionadas
se transforma en un conjunto más pequeño de constructos hipotéticos no correlacionados llamados \textit{componentes principales}.
Su objetivo es condensar la información proporcionada por dichas variables en unas pocas de ellas o en pocas combinaciones lineales de 
ellas (con máxima variabilidad). \newline %referencia PCA romero béjar, % referencia: Bib-5


Las componentes principales se definen como combinaciones lineales de las variables originales que capturan la mayor variabilidad
posible en los datos. Matemáticamente, si $Y$ es un vector de $N$ variables observadas con media $\mu$ y matriz de covarianza $\Sigma$,
las componentes principales $Z_{i}$ se obtienen como:

%duda: ¿tengo que explicar que suponemos normalidad multivariante?

\[
Z_{i} = p^{'}_{i} Y, \hspace{0.2cm} i=1,2,...,N
\]

donde $p_{i}$ es un vector de pesos o \textit{cargas principales} que maximizan la varianza de $Z_{i}$ bajo la restricción de que $p_{i}$
tiene norma unitaria, es decir,

\[%duda: ¿tengo que decir la norma que estoy usando?
\max Var(Z_{i}) = p_{i}^{'}\Sigma p_{i}, \text{ sujeto a } p_{i}^{'}p_{i} = 1.
\]

y tal que asegura que las componentes principales son ortogonales entre sí, es decir:

\[
p_{i}^{'}p_{j} = 0, \text{ para } i \neq j.
\]

Así, garantizamos que las componentes principales $Z_{i}$ y $Z_{j}$ son intercorrelacionadas, es decir, su covarianza es cero para $i\neq j$.


Los vectores $p_{i}$ son los autovectors de la matriz de covarianza $\Sigma$, y los valores propios $\lambda_{i}$, corresponden
a la varianza explicada por cada componente principal. La transformación completa de los datos se expresa de la siguiente forma:

\[
Z=P^{'}Y
\]

donde $P$ es la matriz de autovectores de $\Sigma$, lo que garantiza que las componentes principales sean ortogonales entre sí y no 
correlacionadas, cada una con las anteriores. \newline

Las CP se utilizan para descubrir e interpretar las dependencias que existen entre las variables y para examinar las relaciones
que pueden existir entre los individuos. También son útiles para estabilizar las estimaciones, evaluar la normalidad
multivariante y detectar valores atípicos. \newline

% referencias: bib-5, biib-6, PCA-béjar

\subsection{Análisis factorial}

El objetivo principal del \textit{análisis factorial} (AF) es capturar la realidad de la manera más simple posible, identificando 
unas pocas variables latentes\footnote[8]{Variable latente: variable no observable que se infiere a partir de un conjunto de variables 
observables utilizando un modelo matemático. } que definen esa realidad. Esta técnica multivariante busca explicar el comportamiento de las $N$ 
variables en la matriz de datos $X$ utilizando un número reducido de variables latentes, denominadas \textit{factores}. Lo ideal es que toda 
la información contenida en $X$ pueda ser representada mediante un número menor de factores. Esta técnica busca explicar las correlaciones 
entre las variables mediante la combinación lineal de dichos factores. Así, cada factor es una variable latente que 
influye en las variables observadas, y cuya presencia se infiere a partir de las correlaciones entre ellas. \newline

Matemáticamente, cada variable observada, $x\in \mathbb{R}^{N}$, 
se expresa como una combinación lineal de estos factores, más un término de error específico:

\[
x_{i} = \sum_{l=1}^{k}q_{il}f_{l} + \mu_{i}, \text{ } i=1,...,N.
\]

Aquí, $f_{l}$, con $l=1,...,k$ denota a los factores. El número de factores, $k$, debería ser siempre mucho más pequeño que $N$. \newline

En definitiva, el modelo de análisis factorial asume que la variable observada $x$ puede descomponerse en dos componentes: una parte explicada
por los factores comunes y una parte específica de cada variable. Esto se expresa matricialmente de la siguiente forma:

\[
x=\Lambda F + \psi
\]

donde:

\begin{itemize}
    \item $\Lambda$ es la matriz de cargas factoriales de dimensión $N \times k$,
    \item $F$ es el vector de factores de dimensión $k$.
    \item $\psi$ es el vector de factores específicos o residuales de dimensión $N$
\end{itemize}

El análisis factorial fue desarrollado por Charles Spearman a principios del siglo XX para modelar la inteligencia humana, 
postulando que las puntuaciones en distintas pruebas estaban intercorrelacionadas debido a un único factor latente de inteligencia 
general (g). Su modelo de un solo factor fue posteriormente generalizado por Thurstone a múltiples factores. \newline

El análisis de componentes principales (PCA)  y el análisis factorial suelen confundirse porque ambos analizan 
la variación en un conjunto de variables a partir de la matriz de correlación o covarianza. Sin embargo, mientras que en el EFA unas 
pocas variables latentes explican las correlaciones observadas, en el PCA se necesitan todos los componentes principales para describir 
completamente la variabilidad. Así, el PCA se centra en explicar la varianza total, mientras que el EFA se enfoca en las relaciones 
entre las variables mediante factores comunes. 

%refernecia: Bib-5, Bib-6, AF-romero bejar.


\subsection{Análisis Discriminante}

El \textit{análisis discriminante} es una técnica multivariante que permite identificar un subconjunto de variables y funciones
asociadas que maximicen la separación entre los grupos o poblaciones de estudio. Su objetivo principal es construir funciones
discriminantes que describan y caractericen la separación de los grupos, evaluar el grado de diferenciación y analizar la contribución
de cada variable a la discriminación. \newline

Cuando estas funciones son combinaciones lineales de las variables originales, se denominan funciones discriminantes lineales (LDF). En
particular, el análisis discriminante lineal de Fisher busca encontrar una combinación lineal de variables que maximice la separación
entre grupos. Para dos grupos con medias $\mu_{1}$ y $\mu_{2}$ y una matriiz de covarianza común $\Sigma$, la función discriminante de Fisher
se define como:

\[
L = a^{'}y = \sum_{j=1}^{N} a_{j}y_{j}
\]

donde $a$ es el vector de coeficientes de la función discriminante e $y$ es el vector de observaciones de las variables. \newline

El vector $a$ que maximiza la separación entre los grupos se obtiene como:

\[
a_{\Sigma} = \Sigma^{-1}(\mu_{1}-\mu_{2})
\]

Además, la \textit{distancia de Mahalanobis}, que explicaremos con detalle en la siguiente sección, se emplea para medir la separación entre
los centroides de los grupos:

\[
D^{2} = (\mu_{1}-\mu_{2})^{'}S^{-1}(\mu_{1}-\mu_{2})
\]

Si $D^{2}$ es significativo, implica una buena discriminación entre los grupos. \newline

Este análisis tiene aplicaciones en diversos campos: en biología, Fisher (1936) lo utilizó para diferenciar especies de iris en función de
características morfológicas; en la gestión de personal, permite clasificar profesionales según sus habilidades; en medicina, ayuda a distinguir
entre individuos con alto o bajo riesgo de enfermedades; y en la industria, contribuye a identificar cuándo un proceso está bajo control o fuera 
de control. \newline

Para el caso de múltiples grupos, la función discriminante se construye de manera que maximice la variabilidad entre los grupos en relación con 
la variabilidad dentro de los grupos, lo que se logra mediante una descomposición en valores propios. \newline

Toda la información ha sido extraída de las fuentes bibliográficas\cite{bejar-PCA},\cite{bejar-AF}\cite{Bib-5},\cite{Bib-6} %bib-5, biib-6, PCA-béjar, AF-romero bejar

\section{Análisis Clúster}

% VER REFERENCIAS clustering-1, clustering-2

\subsection{Introducción}

\begin{quote}
    Un ser inteligente no puede tratar cada objeto que ve como una entidad única, diferente de cualquier otra en el universo. Debe categorizar los 
    objetos para poder aplicar el conocimiento adquirido con tanto esfuerzo sobre objetos similares encontrados en el pasado al objeto en cuestión.

\textit{Steven Pinker, Cómo funciona la mente, 1997}
\end{quote}

Una de las habilidades básicas que poseemos los seres humanos es la de agrupar objetos similares para generar una clasificación. Esta idea de 
clasficar cosas similares en categorías es bastante primitiva. Nuestros antepasados prehistóricos debían ser capaces de darse cuenta de que muchos
objetos tenían propiedades semejantes, como ser comestibles, venenosos, peligrosos, etc. \newline

Organizar datos en grupos razonables es uno de los modos más fundamentales de comprensión y aprendizaje. De hecho, es vital para el desarrollo del 
lenguaje, el cual consiste en palabras que nos ayudan a reconocer y analizar los diferentes tipos de eventos, objetos y personas con los que nos
relacionamos. Por ejemplo, los sustantivos en una lengua son palabras que describen una clase de cosas que comparten unas características comunes; 
gatos, perros, caballos, etc., y dicho nombre agrupa a los individuos en grupos. \newline

Además de ser una actividad conceptual humana básica, la clasificación es fundamental en muchas ramas de la ciencia. En biología, por ejemplo, la 
clasificación de organismos (\textit{taxonomía}) ha sido una gran preocupación desde las primeras investigaciones. Aristóteles ya ideó un sistema para clasificar las especies
del reino animal; comenzó dividiendo a los animales en dos grupos: los que tienen sangre roja (vertebrados) y los que carecen de ella (invertebrados). 
Además subdividió estos grupos según su forma de reproducirse: vivas, en huevos, en pupas, etc. Luego, Theophrastos escribió lo relativo a las plantas.
Los libros resultantes estaban tan bien documentados y eran tan completos, tan profundos y de un alcance tan amplio que sentaron las bases de la investigación
biológica duranre siglos. \newline

No fue ya hasta los siglos XVII y XVIII cuando los exploradores europeos crearon un nuevo programa similar de investigación y recolección bajo la dirección del 
sueco Linnaeus, quien estableció un sistema de clasificación que sentó las bases de la taxonomía moderna. Su metodo no solo organizaba a los seres vivos
en categorías jerárquicas, sino que también reflejaba una idea más profunda: la clasificación es esencial para el conocimiento. \newline

En este sentido, todo el conocimiento real que poseemos, depende de métodos con los que podamos distinguir los similar de lo diferente. Cuanto mayor sea el 
número de distinciones naturales que un método comprenda, más clara será nuestra idea de las cosas. A medida que el número de objetos de estudio crece, la 
necesidad de desarrollar sistemas de clasificación más precisos se vuelve aún más necesaria. \newline

En cierto nivel, un esquema de clasificación puede simplemente representar un método conveniente para organizar un gran conjunto de datos, de modo que se pueda
comprender con mayor facilidad y la información se recupere de forma más eficiente. Si los datos pueden resumirse adecuadamente mediante un pequeño número de grupos
de objetos, las etiquetas de grupo pueden proporcionar una descripción muy consisa de los patrones de similitudes y diferencias entre los datos. La necesidad de 
resumir conjuntos de datos de esta manera es cada vez más importante debido al creciente número de grandes bases de datos disponibles en muchas áreas de la ciencia,
como la transcriptómica, que es la que nos ocupa en este trabajo, y la exploración de dichas bases de datos mediante \textit{análisis clúster} y otras técnicas de 
análisis multivariante se denomina hoy día \textit{minería de datos}. \newline

Las técnicas numéricas de clasificación se originaron principalmente en las ciencias naturales, con el objetivo de liberar a la taxonomía de su naturaleza tradicionalmente
subjetva. El objetivo era dar clasificaciones objetivas y estables. Objetivas en el sentido de que el análisis del mismo conjunto de organismos mediante la misma 
secuencia de métodos numéricos produce la misma clasificación; estables en cuanto a que la clasificación permanece igual ante la adición de organismos o de nuevas
características que los describen. \newline

Se le han dado muchos nombres a estas técnicas numéricas, dependiendo del área de aplicación. En biología, el término más extendido es el de \textit{taxonomía numérica}.
En psicología, se usa mucho el término \textit{análisis Q}. En inteligencia articial, el reconocimiento de patrones no supervisado es el término predilecto. Sin embargo,
hoy en día, el \textit{análisis clúster} es probablemente el término genérico para los procedimientos que buscan descubrir grupos en los datos. \newline

La información recogida en esta introducción ha sido extraída de la fuente bibliográfica % clustering-2.pdf

\subsection{¿Qué es el Análisis Cluster (AC)?}

El \textit{análisis cluster} puede definirse como el estudio formal de los algoritmos y métodos de clasificación de objetos. Un objeto es descrito por un conjunto de mediciones o
bien de relaciones entre el objeto y otros objetos. No usa etiquetas de categoría que etiqueten objetos con identificadores previos. A diferencia del análisis discriminante, el 
análisis clúster no utiliza etiquetas predefinidas para clasificar los objetos, sino que busca descubrir estructuras en los datos de manera autónoma. \newline

El objetivo del análisis cluster es agrupar objetos formando conjuntos (clusters) en los que los elementos dentro de cada uno sean lo más similares posible entre sí (baja 
variabilidad interna), mientras que los diferentes grupos sean lo más distintos entre sí (alta variabilidad entre ellos). Es, en definitiva, una técnica exploratoria que 
identifica patrones de similitud dentro de un conjunto de datos, agrupando elementos con características comunes mientras mantiene separadas 
las estructuras con mayores diferencias.\newline % referencia https://digibug.ugr.es/bitstream/handle/10481/85861/AC.pdf?sequence=1&isAllowed=y

Un cluster puede entenderse como un conjunto de elementos que presentan una alta cohesión interna (homogeneidad) y una clara separación externa con respecto a otros grupos. 
Sin embargo, la definición formal de un cluster es difícil de establecer y depende en gran medida del juicio del usuario y del contexto en el que se aplica. Mientras que 
algunos métodos de análisis buscan identificar estructuras naturales en los datos, en muchas ocasiones el proceso de agrupamiento puede imponer una estructura artificial 
en la información. Esto resalta la importancia de interpretar con cautela los resultados de un análisis de clusters, ya que no siempre reflejan patrones inherentes a los 
datos, sino que pueden ser el resultado de los criterios específicos utilizados en la clasificación. \newline %referencia: clustering-2.pdf \newline

\begin{center}
    \includegraphics[width=0.5\textwidth]{../img/cluster-1.png}
\end{center}

En la mayoría de las aplicaciones del AC se busca una partición de los datos en la que cada individuo u objeto pertenezca a un único cluster y el conjunto completo de clusters
contenga a todos los individuos. Sin embargo, esto no siempre es así y, de hecho, en algunas circunstancias, la superposición de clusters puede ofrecer una solución más aceptable.
Decimos que una respuesta aceptable del análisis cluster es que no se justifica la agrupación de los datos. El análisis cluster es un procedimiento objetivo; no están predefinidos,
sino que se forman a medida que avanza el análisis. \newline

Los datos básicos para la mayoría de las aplicaciones del análisis cluster se almacenan en una matriz $n \times p$, $X$, que contiene los valores de las variables que describen
cada objeto que se va a agrupar, es decir,

\[
X = [x_{ij}]_{i=1,\dots,n,\, j=1,\dots,p} =
\begin{bmatrix}
x_{11} & x_{12} & \cdots & x_{1p} \\
x_{21} & x_{22} & \cdots & x_{2p} \\
\vdots & \vdots & \ddots & \vdots \\
x_{n1} & x_{n2} & \cdots & x_{np}
\end{bmatrix}
\]
%       OBSERVA QUE ES LO CONTRARIO DE LO QUE DIJIMOS EN DATOS ÓMICOS (LA TRASPUESTA)?????? QUÉ HAGO?
donde $x_{ij}$ cuantifica la variable $j$ en la muestra $i$. El AC tratará de desarrollar un esquema de clasificación que particionará las filas de $X$ en $k$ clusters. %referencia: clustering-2.pdf


% medidas de proximidad (simularidad, disimilaridad)
\subsection{Medidas de proximidad}

Para desarrollar esta sección se han utlizado las siguientes referencias bibliográficas: \newline%clustering-2 %Bib-5 \newline

Dado que el análisis cluster trata de identificar los vectores que son similares y agruparlos en clusters, es esencial contar con herramientas que permitan evaluar la cercanía o 
distancia entre los objetos que se están agrupando. Las decisiones sobre cómo se van a formar los clusters dependen directamente de las medidas de proximidad utilizadas, ya que estas 
definen cómo se calcula la similitud o diferencia entre los distintos elementos. \newline

Las \textit{medidas de proximidad} se usan para representar la cercanía de dos objetos. Si una medida de proximidad representa \textit{similaridad}, el valor de la medida incrementa 
cuanto más similares sean dos objetos. Alternativamente, si la medida de proximidad representa disimilitud, el valor de la medida disminuye a medida que dos objetos se vuelven más 
parecidos.

\subsubsection{Medidas de Disimilaridad}

Cuando todas las variables registradas son continuas, las proximidades entre los individuos generalmente se cuantifican mediante medidas de disimilaridad o medidas de distancia. 
Por ello, definiremos el concepto de \textit{disimilaridad} análogamente al de distancia.

\begin{definicion}
    Sean $\Omega \subset \mathbb{R}^{n}$ un conjunto de puntos de $\mathbb{R}^{n}$ y $x,y \in \Omega$ dos puntos cualesquiera de dicho conjunto. Entendemos por disimilaridad a toda 
    aplicación $d: \Omega \times \Omega \longrightarrow \mathbb{R}$ que satisface las siguientes propiedades:

    \begin{itemize}
        \item[i] $d(x,y) \ge 0$
        \item[ii] $d(x,y)= 0 \Longleftrightarrow x = y$
        \item[iii] $d(x,y) = d(y,x)$ (simétrica)
        
        Se dice que la disimilaridad es \textit{métrica} si satisface una cuarta propiedad:
        \item[iv] $d(x,y) \leq d(x,z) + d(z,y) \forall z \in \Omega$,
        
        y se dirá \textit{ultramétrica} si es métrica y además cumple:
        \item[v] $d(x,y) \leq \max\{d(x,z),d(y,z)\}$
        

    \end{itemize}
\end{definicion}


A continuación presentamos algunas de las medidas de disimilaridad más usadas. Hemos de hacer notar que estas medidas normalmente se usan para medir cuán próximos están los individuos, 
pero si te quiere medir entre variables, también son válidas. Simplemente habría que trasponer la matriz $X$ y trabajar con ella.

% ejemplos para datos de intervalo, datos binarios ,etc.

% clústering jerárquico
% métodos no jerárquicos

\thispagestyle{empty}
\vspace*{\fill}
\begin{center}
    \large Parte III \\
    \vspace{0.5cm}           
    \LARGE \textbf{FUNDAMENTOS INFORMÁTICOS}
\end{center}
\vspace*{\fill}
\newpage
\setcounter{page}{1}  

\printbibliography

% -------------------------------------------------------------------
% MAINMATTER
% -------------------------------------------------------------------

% Si no deseas reseteo de numeración de capítulos, no es necesario usar \mainmatter
%\mainmatter % Esto no es necesario si ya has configurado la numeración como continua

% Información relevante para la elaboración del trabajo.
%\input{capitulos/documentacion}
%\input{capitulos/recomendaciones}

% Añadir tantos capítulos como sea necesario

%\cleardoublepage\part{Segunda parte}
%\input{capitulos/capitulo-ejemplo}

% -------------------------------------------------------------------
% APPENDIX: Opcional
% -------------------------------------------------------------------

%\appendix % Reinicia la numeración de los capítulos y usa letras para numerarlos
%\pdfbookmark[-1]{Apéndices}{appendix} % Alternativamente podemos agrupar los apéndices con un nuevo \part{Apéndices}

%\input{apendices/apendice-ejemplo}
% Añadir tantos apéndices como sea necesario 

% -------------------------------------------------------------------
% GLOSARIO: Opcional
% -------------------------------------------------------------------

%\input{glosario} 

% -------------------------------------------------------------------
% BACKMATTER
% -------------------------------------------------------------------

%backmatter % Desactiva la numeración de los capítulos
%\pdfbookmark[-1]{Referencias}{BM-Referencias}

% BIBLIOGRAFÍA
%-------------------------------------------------------------------

\end{document}

\documentclass{article}

\begin{document}

\begin{itemize}
    \item Objetivos
    \item Introducción
    \item Datos ómicos
    \begin{itemize}
        \item ¿Qué son los datos ómicos?
        \item Estructura de los datos ómicos
        \item Trasnscriptómica (¿espacial?)
    \end{itemize}
    \item Fundamentos matemáticos
    \begin{itemize}
        \item Técnicas multivariantes (aplicables al tratamiento de datos ómicos)
    \end{itemize}
    \item Fundamentos informáticos
    \begin{itemize}
        \item Estructuras de datos ómicos
        \item Implementación (python/R) de técnicas multivariantes para el manejo de E.D ómicos
        \item Técnicas de aprendizaje automático para la identificación de patrones y clasificación de datos ómicos.
        \item ¿Modelos teniendo en cuenta la información espacial de los datos?
    \end{itemize}
    \item Aplicación
    \item Resultados y conclusiones
\end{itemize}

\end{document}
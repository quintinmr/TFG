\section{Datos ómicos}

A la información cuantitativa y cualitativa obtenida a partir de las tecnologías utilizadas
en las distintas ciencias ómicas, se le denomina \textbf{datos ómicos}. Estos datos abarcan 
información genética (genómica), de expresión génica (transcriptómica), de proteínas (proteómica),
metabolitos (metabolómica) y otras áreas emergentes dentro de las ciencias ómicas.

Una de sus características más relevantes es su \textbf{alta dimensionalidad}, lo que genera conjuntos de datos 
masivos y complejos. Esta naturaleza multidimensional y heterogénea de los datos ómicos presenta 
desafíos significativos en su procesamiento, análisis e interpretación.

Por ello, en el presente trabajo se pretende emplear técnicas de estadística multivariante y aprendizaje
automático para la identificación de patrones biológicos relevantes en estos datos.

\subsection{Estructura de los datos}

Los datos con los que trabajaremos se caracterizan por tener una estructura parecida. Analizaremos un
conjunto con pocas muestras frente al gran número de características que observaremos sobre ellas. Apreciamos
aquí el carácter de alta dimensionalidad de los datos ómicos. \newline

Las características que analizamos pueden ser de diferentes tipos, como el nivel de fluorescencia, en el caso
de que estemos trabajando con microarrays, como los de ADN, metilación o proteínas, o el número de lecturas 
alineadas obtenidas en procedimientos de secuenciación. Estas características, pueden estar asociadas a un
elemento de análisis o a un conjunto de muestras en un microarray. O bien, la información puede corresponder a
un gen, un exón\footnote{Exón: un exón es una región del genoma que termina dentro de una molécula de ARN mensajero.
Algunos exones son codificantes, es decir, contienen información para fabricar una proteína, mientras que otros 
son no codificantes. Los genes del genoma están formados por exones e intrones.},
una proteína o una región específica del genoma. \newline

Denotaremos por $N$ al número de características observadas, que será un valor relativamente grande, del orden
de miles. Como hemos mencionado anteriormente, estas características se observan sobre un conjunto reducido de 
individuos, del orden de las decenas, en el mejor de los casos. Sea entonces $n$ el número de muestras sobre
las que serán observadas las variables (características). \newline

Por consiguiente, el problema se enmarca dentro del campo de la \textbf{estadística de alta dimensión}. Esta
situación, donde $N$ supera a $n$, contrasta con lo que se observa en los enfoques estadísticos convencionales,
en los cuales suele ocurrir todo lo contrario: el número de muestras es mayor que el de variables. Aunque esta
desigualdad presenta limitaciones, también abre un nuevo campo de investigación con retos que los métodos 
tradicionales no pueden resolver, lo que motiva el desarrollo de nuevos procedimientos que se explorarán más 
adelante. \newline

Las características las almacenaremos en una matriz, que llamaremos \textbf{matriz de expresión}, dada por:

\[
    x = [x_{ij}]_{i,j=1,...,n}
\]

donde $x_{ij}$ cuantifica la característica $i$ en la muestra $j$. \newline

\textbf{Nota.} Observemos que en un contexto estadístico
convencional, la matriz de datos sería la matriz transpuesta de la que vamos a estar utilizando. \newline

En el supuesto de que $x_{ij}$ esté asociado con un microarray de ADN, entonces, mide un nivel de fluorescencia,
tomando valores positivos, aunque pudiera ser que, tras el procesado de los datos, se diera lugar a expresiones 
negativas.

Por su parte, si se tratase de un dato obtenido mediante la técnica de secuenciación RNA-seq, tendríamos conteos; 
número de lecturas cortas que se alinean sobre un gen, exón o una zona genómica concretos. Un mayor número de
lecturas será indicativo de una mayor expresión de dicha característica. \newline

Los valores observados de una característica sobre el conjunto de todas las muestras (una fila de la matriz de
expresión) son, en el ámbito de la transcriptómica, lo que se conoce como \textit{perfil}, o de forma más general,
perfil de expresión. \newline

En la matriz $x$ los valores correspondientes a las diferentes muestras son independientes entre sí, aunque pueden
haber sido obtenidos bajo condiciones distintas. Por lo tanto, no se trata de réplicas de una misma condición
experimental, sino de observaciones independientes. Es decir, presentan independencia condicional. Sin embargo, las
filas de $x$ representan vectores que sí están relacionados. Por ejemplo, en una matriz de expresión génica, los 
valores de expresión de las filas no son independientes, debido a que los genes tienden a actuar de manera coordinada. \newline

Por lo general, los datos en las columnas de la matriz $x$, no pueden compararse directamente entre sí, por la presencia
de diversos artefactos técnicos y ruido en la medición de la característica de interés. Es por ello que se han desarrollado
técnicas para corregir estos problemas, denominadas como \textbf{técnicas de preprocesado}. Al aplicar estos métodos, los
datos dejan de ser completamente independientes. No obstante, en la mayoría de los estudios este aspecto no se suele
tener en cuenta. Tras la normalización, los datos siguen considerandose independientes por columnas (muestras) y dependientes
por filas. \newline


A la información o variables que describen y caracterizan a las muestras, las llamaremos \textbf{metadatos} o 
\textbf{variables fenotípicas}. En este contexto, el uso de este término es adecuado porque estas variables reflejan
atributos medibles y observables de las muestras, lo que se conoce, en el ámbito de la biología, como \textit{fenotipo}. 
Normalmente tendremos varias variables fenotípicas. 
Llamaremos $y = (y_{1},...,y_{n})$ a los valores observados de una variable en las $n$ muestras. Uno de los casos más típicos
de variable fenotípica es cuando se tienen dos grupos de muestras: casos (individuos que tienen la enfermedad) y controles
(no tienen la enfermedad o condición de interés). En este caso tendríamos $y_{i} = 1$, para un caso e $y_{i} = 0$, si es control.
Si tuvieramos la situación en la que hay $k$ grupos a comparar, con $k > 2$, entonces se utiliza $y_{i} \in \{1,...,n\}$ 
con $i = 1,...,n$. Hemos de recalcar que los valores $y_{k}$ son arbitrarios y pueden tomar cualquier otro par de valores. \newline




% toda la información obtenida del libro "G. Ayala - Bioinformática Estadística (2023).pdf"
% definición de axon: https://www.genome.gov/genetics-glossary/Exon
% definición de fenotipo: https://www.genome.gov/es/genetics-glossary/Fenotipo 
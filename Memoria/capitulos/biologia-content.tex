\newpage
\thispagestyle{empty}
\vspace*{\fill}
\begin{center}
    \large Parte I \\
    \vspace{0.5cm}           
    \LARGE \textbf{DATOS ÓMICOS}
\end{center}
\vspace*{\fill}
\newpage
\setcounter{page}{1}  % Opcional: reiniciar la numeración de páginas si lo necesitas

\newpage

\chapter{Datos ómicos}
\setcounter{page}{\value{page}}  % Esto asegura que no se reinicie el contador de páginas


A la información cuantitativa y cualitativa obtenida a partir de las tecnologías utilizadas
en las distintas ciencias ómicas, se le denomina \textit{datos ómicos}. Estos datos abarcan 
información genética (genómica), de expresión génica (transcriptómica), de proteínas (proteómica),
metabolitos (metabolómica) y otras áreas emergentes dentro de las ciencias ómicas.

Una de sus características más relevantes es su \textit{alta dimensionalidad}, lo que genera conjuntos de datos 
masivos y complejos. Esta naturaleza multidimensional y heterogénea de los datos ómicos presenta 
desafíos significativos en su procesamiento, análisis e interpretación.

En este capítulo, se presenta la estructura de los datos ómicos, destacando su naturaleza matricial en la que 
el número de características (variables) suele superar ampliamente el número de muestras. Se analiza el desafío 
estadístico que representan los datos de alta dimensión y se introduce el problema de la expresión diferencial. 
Además, se aborda la transcriptómica y las tecnologías de secuenciación de ARN (RNA-seq y scRNA-seq), describiendo 
la estructura de los datos que generan y los retos asociados a su manejo, dado su gran volumen y alta dimensionalidad.

La información presentada en las próximas dos secciones ha sido extraída de la fuente bibliográfica\cite{Ayala2023}.

\section{Estructura de los datos}

Los datos con los que trabajaremos se caracterizan por tener una estructura parecida. Analizaremos un
conjunto con pocas muestras frente al gran número de características que observaremos sobre ellas. Apreciamos
aquí el carácter de alta dimensionalidad de los datos ómicos. \newline

Las características que analizamos pueden ser de diferentes tipos, como el nivel de fluorescencia, en el caso
de que estemos trabajando con microarrays \footnote[1]{Microarray: La tecnología de microarrays permite estudiar 
la expresión de múltiples genes simultáneamente. Consiste en fijar miles de secuencias génicas en un chip de vidrio. 
Al exponer una muestra de ADN o ARN, el apareamiento de bases complementarias genera una señal luminosa medible, 
indicando los genes expresados en la muestra\cite{microarray-definition}.}, como los de ADN, metilación o proteínas, o el número de lecturas 
alineadas obtenidas en procedimientos de secuenciación. Estas características, pueden estar asociadas a un
elemento de análisis o a un conjunto de muestras en un microarray. O bien, la información puede corresponder a
un gen, un exón\footnote[2]{Exón: un exón es una región del genoma que termina dentro de una molécula de ARN mensajero.
Algunos exones son codificantes, es decir, contienen información para fabricar una proteína, mientras que otros 
son no codificantes. Los genes del genoma están formados por exones e intrones, que son trozos muy grandes de ARN 
dentro de una molécula de ARN mensajero que interfieren con el código de los exones. 
Estos intrones se eliminan de la molécula de ARN para dejar una serie de exones unidos entre sí de manera que se 
puedan codificar los aminoácidos correctos\cite{exon-definition}\cite{intron-definition}.}, una proteína o una región específica del genoma. \newline

Denotaremos por $N$ al número de características observadas, que será un valor relativamente grande, del orden
de miles. Como hemos mencionado anteriormente, estas características se observan sobre un conjunto reducido de 
individuos, del orden de las decenas, en el mejor de los casos. Sea entonces $n$ el número de muestras sobre
las que serán observadas las variables (características). \newline

Por consiguiente, el problema se enmarca dentro del campo de la estadística de alta dimensión. Esta
situación, donde $N$ supera a $n$, contrasta con lo que se observa en los enfoques estadísticos convencionales,
en los cuales suele ocurrir todo lo contrario: el número de muestras es mayor que el de variables. Aunque esta
desigualdad presenta limitaciones, también abre un nuevo campo de investigación con retos que los métodos 
tradicionales no pueden resolver, lo que motiva el desarrollo de nuevos procedimientos que se explorarán más 
adelante. \newline

Las características las almacenaremos en una matriz, que llamaremos \textit{matriz de expresión}, dada por:

\[
x = [x_{ij}]_{i=1,\dots,N,\, j=1,\dots,n} =
\begin{bmatrix}
x_{11} & x_{12} & \cdots & x_{1n} \\
x_{21} & x_{22} & \cdots & x_{2n} \\
\vdots & \vdots & \ddots & \vdots \\
x_{N1} & x_{N2} & \cdots & x_{Nn}
\end{bmatrix}
\]


donde $x_{ij}$ cuantifica la característica $i$ en la muestra $j$. \newline

\textbf{Nota.} Observemos que en un contexto estadístico
convencional, la matriz de datos sería la matriz transpuesta de la que vamos a estar utilizando. \newline

En el supuesto de que $x_{ij}$ esté asociado con un microarray de ADN, entonces, mide un nivel de fluorescencia,
tomando valores positivos, aunque pudiera ser que, tras el procesado de los datos, se diera lugar a expresiones 
negativas. Por su parte, si se tratase de un dato obtenido mediante la técnica de secuenciación RNA-seq, que será introducida a continuación, tendríamos conteos; 
número de lecturas cortas que se alinean sobre un gen, exón o una zona genómica concretos. Un mayor número de
lecturas será indicativo de una mayor expresión de dicha característica. \newline

Los valores observados de una característica sobre el conjunto de todas las muestras (una fila de la matriz de
expresión) son, en el ámbito de la transcriptómica, lo que se conoce como \textit{perfil}, o de forma más general,
perfil de expresión. \newline

En la matriz $x$ los valores correspondientes a las diferentes muestras son independientes entre sí, aunque pueden
haber sido obtenidos bajo condiciones distintas. Por lo tanto, no se trata de réplicas de una misma condición
experimental, sino de observaciones independientes. Es decir, presentan independencia condicional. Sin embargo, las
filas de $x$ representan vectores que sí están relacionados. Por ejemplo, en una matriz de expresión génica\footnote[5]{Expresión génica: 
La expresión génica es el proceso por el cual la información codificada por un gen se usa 
para producir moléculas de ARN que codifican para proteínas o para producir moléculas de ARN no codificantes que cumplen otras funciones. La expresión génica
actúa como un “interruptor” que controla cuándo y dónde se producen moléculas de ARN y proteínas y como un “control de volumen” para determinar qué cantidad
de esos materiales se produce\cite{expresion-genica-definition}.}, los 
valores de expresión de las filas no son independientes, debido a que los genes tienden a actuar de manera coordinada. \newline

Por lo general, los datos en las columnas de la matriz $x$, no pueden compararse directamente entre sí, por la presencia
de diversos artefactos técnicos y ruido en la medición de la característica de interés. Es por ello que se han desarrollado
técnicas para corregir estos problemas, denominadas como \textit{técnicas de preprocesado}. Al aplicar estos métodos, los
datos dejan de ser completamente independientes. No obstante, en la mayoría de los estudios este aspecto no se suele
tener en cuenta. Tras la normalización, los datos siguen considerandose independientes por columnas (muestras) y dependientes
por filas. \newline

A la información o variables que describen y caracterizan a las muestras, las llamaremos \textit{metadatos} o 
\textit{variables fenotípicas}. En este contexto, el uso de este término es adecuado porque estas variables reflejan
atributos medibles y observables de las muestras, lo que se conoce en el ámbito de la biología como \textit{fenotipo}\cite{phenotype-definition}. 
Normalmente tendremos varias variables fenotípicas. 
Llamaremos $y = (y_{1},...,y_{n})$ a los valores observados de una variable en las $n$ muestras. Uno de los casos más típicos
de variable fenotípica es cuando se tienen dos grupos de muestras: casos (individuos que tienen la enfermedad) y controles
(no tienen la enfermedad o condición de interés). En este caso tendríamos $y_{i} = 1$, para un caso e $y_{i} = 0$, si es control.
Si tuvieramos la situación en la que hay $k$ grupos a comparar, con $k > 2$, entonces se utiliza $y_{i} \in \{1,...,n\}$ 
con $i = 1,...,n$. Hemos de recalcar que los valores $y_{k}$ son arbitrarios y pueden tomar cualquier otro par de valores.\newline


% toda la información obtenida del libro "G. Ayala - Bioinformática Estadística (2023).pdf"
% definición de microarray: https://www.genome.gov/es/genetics-glossary/Tecnolog%C3%ADa-de-microarrays-chips-de-ADN-o-ARN 
% definición de axon: https://www.genome.gov/genetics-glossary/Exon
% definición de intrón: https://www.genome.gov/es/genetics-glossary/Intron 
% definición de fenotipo: https://www.genome.gov/es/genetics-glossary/Fenotipo 


\section{Problema Estadístico}

Normalmente, las técnicas estadísticas utilizadas en muchos campos se basan en contextos en los que 
el número de muestras, $n$, es mayor que el de variables, $N$. Sin embargo, en el caso de los datos ómicos, esta relación se invierte,
lo que obliga a ajustar estos procedimientos de manera que, en algunos casos, la adaptación resulta más o menos exitosa. En otras palabras,
la falta de suficientes muestras para la cantidad de variables presentes, hace que sea extremadamente difícil encontrar un modelo que pueda capturar
de manera precisa la relación entre las variables predictores y la variable respuesta. Esto se debe a que no tenemos una cantidad adecuada 
de datos para entrenar de manera efectiva un modelo estadístico que pueda generalizarse de manera fiable a nuevas observaciones.

La dificultad de analizar datos de alta dimensionalidad resulta además de la conjunción de dos efectos.

En primer lugar, los espacios de alta dimensión tienen propiedades geométricas que son contra-intuitivas y alejadas de las propiedades
que se pueden observar en espacios bidimensionales o tridimensionales. 

En segundo lugar, las herramientas de análisis de datos suelen diseñarse teniendo en cuenta propiedades intuitivas
y ejemplos en espacios de baja dimensión; por lo general, las herramientas de análisis de datos se ilustran mejor en espacios de dos y 
tres dimensiones, por razones obvias. El problema es que esas herramientas también se utilizan cuando los datos son de alta dimensión y
más complejos. En este tipo de situaciones, perdemos la intuición del comportamiento de las herramientas y podemos sacar conclusiones erróneas
sobre sus resultados, dificultando la construcción de modelos estadísticos precisos. \newline

Por todo ello, en este contexto, las técnicas estadísticas utilizadas son mera aplicación de procedimientos diseñados para la situación antes comentada en la 
que el número de muestras es mayor que el de variables. \newline

Uno de los principales retos que se abordarán es el análisis de expresión diferencial, 
que examina cómo varían los niveles de expresión génica entre distintas condiciones experimentales o grupos de individuos. En particular, se
busca determinar si existe una relación entre el perfil de expresión génica y una variable fenotípica específica. Este enfoque, denominado análisis
de expresión diferencial marginal, permite explorar asociaciones entre conjuntos de genes, organizados como grupos de filas dentro de la matriz de expresión, 
y la característica fenotípica de interés. A este tipo de análisis se le conoce también como análisis de conjuntos de genes o \textit{gene set analysis}, y 
su objetivo es identificar patrones de expresión que puedan estar vinculados a determinados rasgos biológicos. \newline

% https://www.genome.gov/es/genetics-glossary/Expresion-genica -- Definición expresión génica
% toda la información obtenida del libro "G. Ayala - Bioinformática Estadística (2023).pdf"


\section{Transcriptómica: datos RNA-seq y single-cell RNA-seq}

Entre las distintas tecnologías utilizadas en la generación de datos ómicos, la \textit{transcriptómica} desempeña un papel fundamental en el análisis de la expresión génica, y en particular, las 
tecnologías de \textit{RNA-seq} y \textit{single-cell RNA-seq} han revolucionado este campo al permitir la cuantificación precisa de los 
niveles de ARN mensajero en diferentes condiciones biológicas. Estas técnicas producen datos con una estructura matricial 
compleja, caracterizada por un alto número de variables (genes) frente a un número reducido de muestras (individuos o células). 
Esta estructura plantea desafíos en términos de almacenamiento, procesamiento y análisis, debido a la alta dimensionalidad de 
los datos generados. Nos centraremos en la transcriptómica por su capacidad para ofrecer una visión dinámica 
y profunda de la actividad celular, permitiendo identificar patrones de expresión génica que reflejan procesos biológicos clave. \newline

En este apartado, se describirá la estructura de los datos obtenidos mediante RNA-seq y single-cell RNA-seq, y se discutirán los 
principales retos asociados a su manejo, desde las consideraciones técnicas hasta las implicaciones estadísticas y computacionales, 
que hacen de la transcriptómica un área idónea para aplicar metodologías multivariantes en la identificación de patrones biológicos.

\subsection{¿Qué es la transcriptómica? Tecnologias de secuenciación}

La transcriptómica es la rama de la biología que estudia el conjunto completo de ARN (ácido ribonucleico) transcritos en una
célula, tejido u organismo en un momento determinado, bajo condiciones específicas. A este conjunto se le denomina transcriptoma. Se centra en la cuantificación y caracterización
de los distintos tipos de ARN, incluyendo ARN mensajero (mARN), ARN de transferencia (tRNA), ARN ribosomal (rRNA) y ARN 
no codificante (ncRNA), entre otros. Este campo ha evolucionado significativamente desde la formulación del dogma central de la biología molecular por Francis Crick
en 1958, que estableció la transferencia de información genética desde el ADN al ARN y posteriomennte a las proteínas. \newline

A medida que la transcriptómica ha avanzado, se han ido desarrollando varias tecnologías para deducir y cuantificar el transcriptoma, basadas
tanto en hibridación como en secuenciación. Los enfoques basados en hibridación, como los microarrays, pese a que son más económicos y 
tienen un alto rendimiento, dependen del conocimiento existente sobre la sencuencia del genoma. \newline

A diferencia de los métodos basados en microarrays, los enfoques basados en secuencias determinan directamente la secuencia de ARN. Inicialmente, se utilizó
la secuenciación de Sanger de bibliotecas de ARN, pero era bastante costosa y de bajo rendimiento y generalmente no cuantitativa. Se 
desarrollaron métodos basados en etiquetas para superar estas limitaciones, pero tenían el inconveniente de que estaban basados en la 
costosa tecnolgía de secuenciación de Sanger, y una parte significativa de las etiquetas cortas no se podían asignar de forma única al
genoma de referencia. Todas estas desventajas limitan el uso de la tecnología de secuenciación tradicional para anotar la estructura de los 
transcriptomas\cite{transcriptomics-1}. \newline    

% estos 3 párrafos sacados del archivo transcriptomics-1.pdf

Tecnologías de secuenciación de ADN de alto rendimiento como \textit{RNA-seq} y \textit{single-cell RNA-seq} han emergido como herramientas 
clave para estudiar la expresión génica a gran escala. Estas técnicas permiten la cuantificación precisa de los niveles de ARN en diferentes 
condiciones biológicas, a diferencia de los métodos basados en microarrays, lo que proporciona información valiosa sobre la actividad celular 
y los mecanismos de regulación genética. Sin embargo, los datos obtenidos mediante RNA-seq y single-cell RNA-seq poseen características 
específicas que influyen en su representación y análisis. Estas tecnologías generan grandes volúmenes de datos con una estructura 
matricial compleja, en la que el número de características (genes) supera ampliamente al número de muestras, lo que da lugar a retos 
significativos en términos de almacenamiento, procesamiento y análisis.

\subsection{RNA-seq}

El método RNA-seq (secuenciación de ARN) consiste en la conversión de una muestra de ARN (total o fraccionado) en una biblioteca de ADNc 
(ADN codificado), que luego es secuenciada utilizando tecnologías de secuenciación profunda. Genera un conjunto masivo de datos que consiste 
en lecturas cortas de ARN transcrito, secuencias que generalmente varían entre 30 y 400 pares de bases de longitud, representando fragmentos 
de transcritos provenientes de ARN mensajero (ARNm) o ARN no codificante. Estas secuencias se alinean con un genoma de referencia o con 
transcritos de referencia para mapear la estructura transcripcional y cuantificar la expresión génica. Una de las principales aplicaciones 
de RNA-seq es el análisis de expresión diferencial, mencionado en la sección previa y que abordaremos de forma práctica, que permite comparar 
los niveles de expresión de genes entre diferentes condiciones 
biológicas, como células tratadas frente a no tratadas, tejidos sanos frente a cancerosos o distintos estados del desarrollo. Esto ofrece 
una visión detallada de los cambios en la actividad génica y ayuda a identificar biomarcadores, rutas metabólicas alteradas o procesos 
reguladores clave. Además, RNA-seq no está limitado a detectar solo transcritos que corresponden a secuencias genómicas conocidas, 
lo que lo hace particularmente útil para organismos no modelo o cuando se carece de un genoma de referencia bien caracterizado\cite{transcriptomics-2-RNA-seq}.\newline
% REFERENCIA: RNA-seq: A revolutionary tool....

En la secuenciación de ARN, las lecturas generadas a partir de las muestras de ARN se alinean contra un genoma de referencia o se ensamblan de
nuevo para crear un \textit{mapa transcripcional}. Si se dispone de un genoma de referencia, los datos se alinean para identificar la ubicación exacta de
los transcritos en el genoma, permitiendo la cuantificación de la expresión génica. En casos donde no hay un genoma de referencia, las lecturas 
de ARN se ensamblan para generar una secuencia de contigs\footnote[3]{Contig: Tramo de secuencia continua in silico generada por alineamiento 
de lecturas de secuencias solapantes\cite{contig-definition}.} que luego se pueden anotar funcionalmente. Además de mapear las lecturas a un genoma, 
se deben identificar eventos de empalme (splicing\footnote[4]{Splicing: el splicing o empalme, ocurre al final del proceso de transcripción e 
implica cortar y reorganizar secciones de ARNm\cite{splicing-definition}.}), que es crucial para detectar variantes de splicing alternativo. Este proceso es especialmente 
importante para genes que tienen varios exones, ya que las lecturas pueden cruzar estos empalmes y revelar alternativas de splicing que no son 
evidentes con tecnologías anteriores\cite{transcriptomics-2-RNA-seq-2}. \newline % REFERENCIA: Mapping and quantifying mammalian transcriptomes by RNA-seq

% contig: https://www.institutoroche.es/recursos/glosario/contig
% splicing: https://www.yourgenome.org/theme/what-is-rna-splicing/

Pese a las ventajas que la RNA-seq tiene frente a tecnologías anteriores, los conjuntos de datos producidos son grandes y complejos y la interpretación
no es sencilla. La interpretación de los datos de secuenciación de ARN depende de la cuestión científica de interés. El objetivo principal de muchos 
estudios biológicos es el perfil de expresión génica entre muestras, que es particularmente relevante, por ejemplo, para experimentos controlados 
que comparan la expresión en cepas de tipo salvaje y mutantes del mismo tejido, comparando células tratadas versus no tratadas, cáncer versus normal, etc\cite{transcriptomics-2-RNA-seq-3}. \newline
% REFERENCIA: transcriptomics-RNA-seq-3

Por otra parte, los datos RNA-seq requieren estar en unos formatos específicos para su tratamiento. Formatos adecuados para almacenar secuencias tanto
de ácidos nucleicos como de proteínas. Estos son: formato \textit{FASTA} y \textit{FASTQ}. La información que presentamos a continuación ha sido extraída de\cite{Ayala2023}.

\begin{itemize}
    \item \textbf{Formato FASTA:} basado en texto, es usado para representar secuencias de nucleótidos o de aminoácidos, ambos representados mediante una sola letra.
    Incluye símbolos para representar huecos (\textit{gaps}) o posiciones desconocidas en la secuencia. Consta de dos líneas:
    \begin{itemize}
        \item La primera línea comienza con el símbolo $>$, junto con una descripción de la secuencia.
        \item La segunda, contiene la secuencia de bases o aminoácidos. 
    \end{itemize}

    Pese a que no hay restricciones en el número de filas, el número de columnas no debería superar las $80$.
    
    \item \textbf{Formato FASTAQ:} es el más popular y consiste en cuatro líneas por lectura:
    \begin{itemize}
        \item La primera comienza con el carácter "@" y contiene el nombre de la secuencia. Opcionalmente, puede incluir una descripción.
        \item La segunda línea contiene la secuencia con las letras correspondientes, dependiendo del tipo de secuencia del que se trate (nucleótido o aminoácido).
        \item La tercera comienza con el carácter "+" y contiene información opcional sobre la secuencia.
        \item La cuarta y última línea cuantifica la calidad o confiabilidad de cada base en la secuencia recogida en la segunda línea, basada en el índice
        \textit{Phred} y su codificación. \newline
    \end{itemize}

    \begin{mybox}{Ejemplo}
        

        \begin{lstlisting}

        @SRR1293399.1 ILLUMINA-545855_0026_FC629BG:6:1:1022:5049 length=50
        ACAGGGACGCCATCGAATCCGGATCNTNNNNNNNNNNNNANNNNNNNNNN
        +SRR1293399.1 ILLUMINA-545855_0026_FC629BG:6:1:1022:5049 length=50
        dee\edYcdc`bbY`S]bb_]]Ua^BBBBBBBBBBBBBBBBBBBBBBBBB
            
        \end{lstlisting}
        
    \end{mybox}

\end{itemize}

\subsection{scRNA-seq (sigle-cell RNA-seq)}

La secuenciación de ARN ha impulsado muchos descubrimientos e innovaciones en diversos campos en los últimos años. Por razones prácticas, la técnica RNA-seq
suele realizarse en muestras que comprenden entre miles y millones de células. Sin embargo, esto ha dificultado la evaluación directa de la unidad fundamental
de la biología: la célula. Es por esto que surge una variante a la RNA-seq: \textit{scRNA-seq} (\textit{single-cell RNA-seq}), que se ha extendido considerablemente hasta 
consolidarse como la opción principal en investigación de la diversidad celular, ya que posibilita el estudio individual de cada célula dentro de una misma muestra. 
Es una tecnología que permite la cuantificación y comparación de los transcriptomas de células individuales, logrando una disección de la expresión génica a resolución de una 
sola célula y describiendo moléculas de ARN con alta precisión y a escala genómica. Además, otro aspecto importante de esta técnica es que permite analizar 
la heterogeneidad celuar; evaluar las similitudes y diferencias transcripcionales dentro de una población de células, lo que ha permitido una comprensión más 
detallada de los procesos moleculares al evidenciar la variabilidad existente dentro de una misma población celular. \newline

Aunque existe una gran confianza en la utilidad general de scRNA-seq, hay una barrera técnica que debe considerarse cuidadosamente: el aislamiento efectivo de 
células individuales del tejido de interés, pues existe un riesgo potencial de que los protocolos utilizados para este proceso alteren los niveles de ARNm. 
Aunque las células vecinas pueden contribuir a mantener los estados celulares, scRNA-seq opera bajo el supuesto de que el aislamiento de células individuales 
no desencadena cambios transcriptómicos antes de la captura del ARNm. Además, esta técnica a menudo requiere el uso de marcadores específicos para distinguir 
con precisión diferentes poblaciones celulares, lo que puede añadir complejidad al diseño experimental. A esto se suman limitaciones relacionadas con el tiempo 
y los costos asociados, que pueden dificultar la realización de investigaciones a gran escala. \newline

El análisis de datos provenientes de scRNA-seq supone un desafío computacional considerable debido a su alta dimensionalidad y a la presencia de ruido significativo. 
En el estudio de muestras heterogéneas, es fundamental identificar las distintas subpoblaciones celulares presentes, lo que requiere el desarrollo de algoritmos 
capaces de descomponer la mezcla de expresión génica obtenida. Para ello, se aplican técnicas de normalización y reducción de dimensionalidad, como el análisis 
de componentes principales (PCA), entre otras, que permiten segmentar las células en función de sus perfiles de 
expresión. No obstante, la presencia de efectos por lote y la variabilidad técnica pueden introducir sesgos en los resultados, lo que hace necesario implementar 
estrategias de corrección para garantizar la robustez del análisis y evitar la aparición de artefactos en la identificación de patrones biológicos. \newline

Además, el volumen de datos generado por scRNA-seq plantea retos en términos de almacenamiento, gestión y escalabilidad computacional. La integración de 
millones de lecturas individuales requiere estrategias eficientes de compresión y procesamiento distribuido, así como el uso de infraestructuras de alto 
rendimiento para análisis en estudios de gran escala. Dado que el aislamiento manual de células en el laboratorio puede ser costoso y técnicamente desafiante, 
se han propuesto enfoques computacionales que permitan inferir la composición celular sin necesidad de manipulación experimental. En este contexto, 
los métodos multivariantes desempeñan un papel clave en la extracción de información relevante, facilitando la identificación de estructuras en los datos 
sin necesidad de un conocimiento previo sobre los tipos celulares presentes en la muestra. La combinación de técnicas estadísticas y aprendizaje automático 
ha demostrado ser esencial para mejorar la precisión y fiabilidad de estos análisis en el ámbito biomolecular. \newline

% REFERENCIA: todo se ha sacado de: transcriptomica-3-scRNA-seq-1.pdf

Dado que scRNA-seq es una variante de RNA-seq, hereda sus formatos de almacenamiento iniciales, principalmente FASTA y FASTQ, siendo este último el formato 
más crudo en el que se encuentran los datos de scRNA-seq. % REFERENCIA: https://broadinstitute.github.io/2019_scWorkshop/processing-scrnaseq-data.html
% y también de transcriptomica-3-scRNA-seq-2.pdf 

Toda la información recogida en esta subsección ha sido extraída de las fuentes bibliográficas\cite{transcriptomics-3-scRNA-seq-1},\cite{transcriptomics-3-scRNA-seq-2},\cite{BroadInstitute2019}.
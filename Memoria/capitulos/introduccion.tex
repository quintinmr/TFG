\chapter{Introducción}

La biología, como disciplina científica, ha experimentado una evolución notable en las últimas décadas, 
pasando de enfoques cualitativos y descriptivos a un análisis más detallado y cuantitativo de los organismos 
vivos. Este cambio de paradigma se produjo a mediados del siglo XX con la llegada de la \textit{biología molecular},
tras casi dos siglos de preeminencia del naturalismo basado en la observación y la contemplación. 
Este avance marcó el inicio de una nueva era en la que el desarrollo de ciertas herramientas tecnológicas permitió 
analizar los diversos y complejos niveles de organización de los organismos, generando grandes volúmenes de datos 
en periodos relativamente cortos: la \textit{era de las ciencias ómicas} \cite{referencia-1, referencia-2}. \newline%\cite{referencia1}\cite{referencia2}. \newline

En este contexto, las ciencias ómicas surgieron como un marco integrador que engloba el conocimiento derivado 
de la aplicación de tecnologías avanzadas para el estudio a nivel molecular de los distintos elementos que 
conforman los sistemas biológicos, como células, tejidos e individuos. Estas disciplinas no solo permiten 
analizar la complejidad interna de los organismos, sino también comprender las interacciones dinámicas 
entre sus componentes internos y los factores externos con los que estos interactúan. Ofrecen, en definitiva,
una perspectiva holística del individuo, proporcionando una visión detallada del funcionamiento de sus células 
y de la influencia del entorno que las rodea \cite{referencia-2}. \newline% \cite{referencia2}. \newline

El término \textit{ómica} fue acuñado en la década de 1980 para referirse al estudio de 
\textit{conjuntos de moléculas} específicas, como genes (genómica), transcripciones de ARN (transcriptómica), 
proteínas (proteómica) o metabolitos (metabolómica), entre otros. Estas disciplinas han evolucionado 
significativamente gracias a los avances tecnológicos que permiten abordar la complejidad inherente de los 
sistemas biológicos analizados. De hecho, este es el máximo distintivo de las ciencias ómicas: el uso de las 
llamadas \textit{tecnologías ómicas}, herramientas de alto rendimiento diseñadas para generar grandes cantidades 
de datos en un solo experimento a partir de una única muestra. Este enfoque masivo en la obtención de datos, 
conocido como \textit{Big Data}, ha transformado profundamente el análisis biológico, permitiendo explorar 
dinámicas moleculares con un gran nivel de detalle \cite{referencia-2}\cite{referencia-5}.\newline% \cite{referencia2} \cite{referencia5}. \newline

La integración de las ciencias ómicas con metodologías avanzadas de análisis, como las técnicas multivariantes 
y el aprendizaje automático, ha marcado un hito en la investigación biomédica, abriendo nuevas fronteras en la 
comprensión de los complejos sistemas biológicos. Estas metodologías, que permiten gestionar y analizar grandes 
volúmenes de datos con múltiples dimensiones, son fundamentales para descubrir patrones biológicos subyacentes 
que, de otro modo, podrían pasar desapercibidos utilizando métodos tradicionales \cite{tarca}. Técnicas multivariantes, como 
el análisis de componentes principales (PCA), el análisis clúster, el análisis factorial o el análisis discriminante, 
facilitan la identificación de relaciones y la reducción de la dimensionalidad en los datos, lo que es crucial para 
poder extraer información relevante de los voluminosos conjuntos de datos generados \cite{intro-1}.\newline

A medida que los volúmenes de datos generados por las tecnologías ómicas se incrementan, la \textit{bioinformática} se ha 
consolidado como una disciplina esencial para procesar, gestionar y analizar dichos datos \cite{kaneshia}. Facilita la identificación y 
visualización de patrones biológicos complejos a partir de grandes bases de datos, mediante el uso de algoritmos avanzados,
herramientas computacionales y modelos estadísticos. Este enfoque es fundamental para descubrir asociaciones moleculares, 
determinar biomarcadores relevantes y comprender las bases genéticas de enfermedades \cite{koh}. En este sentido, las herramientas
bioinformáticas, como los lenguajes de programación R y Python, entre otros, junto con plataformas especializadas como Bioconductor, 
permiten realizar análisis profundos de datos ómicos a gran escala, proporcionando los recursos necesarios para un 
manejo efectivo y preciso de la información biológica \cite{bioconductor-1}. \newline

Además, la combinación de las ciencias ómicas con estas metodologías avanzadas ha mejorado significativamente 
nuestra comprensión de los procesos biológicos y ha facilitado el desarrollo de estrategias diagnósticas y terapéuticas 
innovadoras. En particular, las técnicas multivariantes y el aprendizaje automático han demostrado ser esenciales 
para la identificación de patrones biológicos en diversas enfermedades, desde el cáncer hasta trastornos neurodegenerativos, 
así como para predecir la respuesta a distintos tratamientos. Esta integración ha impulsado el avance hacia la medicina 
personalizada y de precisión, en la que los tratamientos se ajustan a las características individuales de cada paciente, 
haciéndolos mucho más eficientes y reduciendo efectos adversos. \newline

En el presente trabajo, se explorará el uso de la transcriptómica, como ciencia ómica y las metodologías avanzadas de análisis de 
datos, como las técnicas multivariantes y el aprendizaje automático, para la identificación y clasificación 
de ciertos patrones biológicos. Se realizará una revisión teórica de las técnicas multivariantes más comunes, 
anteriormente mencionadas, aunque nos centraremos en una de ellas con el fin de proporcionar una base sólida 
para su aplicación práctica en datos ómicos. Posteriormente, se llevará a cabo una implementación realista y 
funcional para el análisis de datos biológicos, aplicando técnicas de aprendizaje automático para la identificación 
de patrones biológicos significativos. A través de estas metodologías avanzadas, se intentará simplificar los datos 
ómicos para poder extraer la información clave que permita clasificar y entender mejor los patrones biológicos, 
mejorando así la precisión de los modelos predictivos.

